%\usepackage[T1]{fontenc}
\usepackage[latin9]{inputenc}
\usepackage[letterpaper]{geometry}
\geometry{verbose,tmargin=1in,bmargin=1in,lmargin=1in,rmargin=1in}
\usepackage{babel}
\usepackage{amsmath}
\usepackage{amssymb}
\usepackage{capt-of}
\usepackage{graphicx}
\usepackage{color}
\usepackage{latexsym}
\usepackage{xspace}
\usepackage{pdflscape}
\usepackage[hyphens]{url}
\usepackage[colorlinks]{hyperref}
\usepackage{enumerate}
\usepackage{ifthen}
\usepackage{float}
\usepackage{array}
\usepackage{tikz}
\usepackage{multirow} 

\usetikzlibrary{shapes}
\usepackage{algorithm2e}

%%%% HW instructions / collaboration text

\newcommand{\HWPolicies}{
\paragraph*{Instructions.} {\bf This homework contains two parts. Part I is written assignment and Part II is MATLAB programming assignment.}

For the written part, please write up your responses to the problem clearly and
concisely. We encourage you to write up your responses using \LaTeX{} and
you may use the short \LaTeX{} template for homework 1. Hand your written homework to Charity Payne in {\bf Levine 459}. Write your name, the time at which you are handing in
  your homework, and the name(s) of any collaborator(s) on the front
  page of your problem set. If the office is closed, please slide the
  homeworks under the door.
  
For the programming part, you can submit your code to be automatically checked for
correctness to receive feedback ahead of time. We are providing you with codebase / templates / dataset on Piazza. {\bf Please read through the documentation provided in ALL Matlab files before
  starting the assignment.} The instructions for submitting your
homeworks and receiving automatic feedback are online on the wiki:

\begin{center}
\url{http://alliance.seas.upenn.edu/~cis520/wiki/index.php?n=Resources.HomeworkSubmission}
\end{center}
}

\newcommand{\ProgrammingPolicies}[1]{
\paragraph*{Instructions.} 
This is a MATLAB programming assignment. This assignment consists of
multiple parts. Portions of each part will be graded automatically,
and you can submit your code to be automatically checked for
correctness to receive feedback ahead of time.

We are providing you with codebase / templates / dataset that you will
require for this assignment. Download the file {\tt hw#1\_kit.zip}
from Blackboard {\bf before} beginning the assignment.  {\bf Please
  read through the documentation provided in ALL Matlab files before
  starting the assignment.} The instructions for submitting your
homeworks and receiving automatic feedback are online on the wiki:

\begin{center}
\url{http://alliance.seas.upenn.edu/~cis520/wiki/index.php?n=Resources.HomeworkSubmission}
\end{center}

\paragraph*{Collaboration.} 
For this programming assignment, you are {\bf not} allowed to
collaborate with other students. You may {\em discuss} the homework to
understand the problem, but {\bf you are not allowed to share your
code with any other students.} We will be using automatic checking
software to detect blatant copying of other student's assignments, so,
please, don't do it.
}

\newcommand{\HWCollabPolicies}{\paragraph*{Collaboration.} 
You are allowed and encouraged to work together. You may discuss the
homework to understand the problem and reach a solution in groups up
to size {\bf four students.} However, {\em each student must write
  down the solution independently, and without referring to written
  notes from the joint session. {\bf In addition, each student must
    write on the problem set the set of people with whom s/he
    collaborated.}}}

\newcommand{\PRCollabPolicies}{\paragraph*{Collaboration.} 
For this programming assignment, you are {\bf not} allowed to
collaborate with other students. You may {\em discuss} the homework to
understand the problem, but {\bf you are not allowed to share your
code with any other students.} We will be using automatic checking
software to detect blatant copying of other student's assignments, so,
please, don't do it.}

%%%% CUSTOM COMMANDS FOR FORMATTING EXAMS/HOMEWORKS
\newcounter{section_points}[section]
\newcounter{header_points}[section]
\newcounter{total_points}

\newcommand{\hpoints}[1]{
  \setcounter{header_points}{#1}
  \textbf{[#1 points]}
}

\newcommand{\points}[1]{
  \addtocounter{section_points}{#1}
  \addtocounter{total_points}{#1}
  \textbf{[#1 
    \ifthenelse{\equal{#1}{1}}
    {point}{points}]}
}

\newcommand{\point}{\textbf{[1 point]}}

\newboolean{ShowSolutions}
\newcommand{\Mistake}[2]{
  \ifthenelse{\boolean{ShowSolutions}}
  {\paragraph{\bf $\blacksquare$ COMMON MISTAKE #1:} {\sf #2} \bigskip}
  {}
}

\newcommand{\Solution}[2]{
  \ifthenelse{\boolean{ShowSolutions}}
    {
      \paragraph{\bf $\bigstar $ SOLUTION:} { \sf
        #1} \bigskip
    }
    { 
      #2
    } %} \vspace{1.5in}}
}
\newcommand{\out}[1]{}

\newboolean{ShowPointsInfo}

\newcommand{\PointStats}[0]{
  \ifthenelse{\boolean{ShowPointsInfo}}
  {
    \begin{center}

      \begin{tabular}{rl}
        \hline
        Stated Points: & \arabic{header_points} \\
        Section Points: & \arabic{section_points} \\
        Total Points So Far: & \arabic{total_points} \\
        \hline 
        \multicolumn{2}{c}{
          
          \ifthenelse{
            \equal{\value{section_points}}{\value{header_points}}
          }{CORRECT TOTAL}
          {{\bf INCORRECT TOTAL}}
          }
          \\
        \hline
      \end{tabular}
    \end{center}
  }{}
}

\newcounter{blankcount}
\newcommand{\myrepeat}[2] {
\setcounter{blankcount}{1}
\whiledo{\value{blankcount} < #1}{
#2
\addtocounter{blankcount}{1}
}
}

\newcommand{\blank}[1]{\underline{\myrepeat{#1}{\qquad}}}


%%%% CUSTOM MATH GOES HERE

\newcommand{\ind}[1]{\mathbf{1}\left(#1\right)}
\renewcommand{\Pr}{\mathbf{Pr}\xspace}
\newcommand{\Bern}{\textsf{Bernoulli}\xspace}
\newcommand{\sign}{\textsf{sign}}

\newcommand{\E}{\mathbf{E}}
\newcommand{\bx}{\mathbf{x}}
\newcommand{\bX}{\mathbf{X}}
\newcommand{\by}{\mathbf{y}}
\newcommand{\bY}{\mathbf{Y}}
\newcommand{\bz}{\mathbf{z}}
\newcommand{\bw}{\mathbf{w}}
\newcommand{\bl}{\mathbf{\ell}}
\newcommand{\vc}[1]{\mathbf{#1}}

\newcommand{\Hypo}{\mathcal{H}}
\newcommand{\XX}{\mathcal{X}}
\newcommand{\cD}{\mathcal{D}}

\newcommand{\argmax}{\operatornamewithlimits{argmax}}
\newcommand{\argmin}{\operatornamewithlimits{argmin}}

\newcolumntype{M}{>{$\vcenter\bgroup\hbox\bgroup}c<{\egroup\egroup$}}




