%\usepackage[T1]{fontenc}
\usepackage[latin9]{inputenc}
\usepackage[letterpaper]{geometry}
\geometry{verbose,tmargin=1in,bmargin=1in,lmargin=1in,rmargin=1in}
\usepackage{babel}
\usepackage{capt-of}
\usepackage{graphicx}
\usepackage{color}
\usepackage{latexsym}
\usepackage{xspace}
\usepackage{pdflscape}
\usepackage[hyphens]{url}
\usepackage[colorlinks]{hyperref}
\usepackage{ifthen}
\usepackage{float}
\usepackage{array}
\usepackage{tikz}
\usetikzlibrary{shapes}
\usepackage{algorithm2e}
\usepackage{listings}
\usepackage{fullpage}
\usepackage{amssymb}
\usepackage{amsmath}
\usepackage{amsthm}
\usepackage{latexsym}
\usepackage{graphicx}
\usepackage{color}
\usepackage{url}
\usepackage{enumerate}
\usepackage{multirow}

\usepackage{subcaption}
\usepackage{xspace}
\newcommand{\bx}{\mathbf{x}}
\newcommand{\bw}{\mathbf{w}}
%%%% HW instructions / collaboration text

\newcommand{\HWPoliciesmix}{
\paragraph*{Instructions.} {\bf This homework contains two parts. Part A is written assignment and Part B is MATLAB programming assignment.}
}

\newcommand{\HWPolicies}{
\paragraph*{Instructions.} 
Please write up your responses to the following problems clearly and
concisely. We encourage you to write up your responses using \LaTeX{};
we have provided a \LaTeX{} template, available on Canvas, to
make this easier. {\bf Submit your answers in PDF form to Canvas. We will not accept paper copies of the homework.}

\paragraph*{Collaboration.} 
You are allowed and encouraged to work together. You may discuss the
homework to understand the problem and reach a solution in groups up
to size {\bf two students.} However, {\em each student must write
  down the solution independently, and without referring to written
  notes from the joint session. {\bf In addition, each student must
    write on the problem set the names of the people with whom you
    collaborated.}} You must understand the solution well enough in
order to reconstruct it by yourself. (This is for your own benefit:
you have to take the exams alone.)
}

\newcommand{\HWPoliciesNew}[1]{
\paragraph*{For Problems 1-4} 

\begin{itemize}
\item \textbf{Instructions.} 
Please write up your responses to the following problems clearly and
concisely. We encourage you to write up your responses using \LaTeX{};
we have provided a \LaTeX{} template, available on Canvas, to
make this easier. {\bf Submit your answers in PDF form to Canvas. We will not accept paper copies of the homework.} 

\item \textbf{Collaboration Policy.} 
You are allowed and encouraged to work together. You may discuss the
homework to understand the problem and reach a solution in groups up
to size {\bf two students.} However, {\em each student must write
  down the solution independently, and without referring to written
  notes from the joint session. {\bf In addition, each student must
    write on the problem set the names of the people with whom you
    collaborated.}} You must understand the solution well enough in
order to reconstruct it by yourself. (This is for your own benefit:
you have to take the exams alone.)

\end{itemize}

\paragraph*{For Problem 5} 


\begin{itemize}

\item
\textbf{Instructions.} 
This is a MATLAB programming problem. You can submit your code to be automatically checked for
correctness to receive feedback ahead of time.

We are providing you with codebase / templates / dataset that you will
require for this problem. Download the file {\tt hw#1\_kit.zip}
from Canvas {\bf before} beginning the assignment.  {\bf Please
  read through the documentation provided in ALL Matlab files before
  starting the assignment.} The instructions for submitting your
homeworks and receiving automatic feedback are online on the wiki:

\begin{center}
\url{http://alliance.seas.upenn.edu/~cis520/wiki/index.php?n=Resources.HomeworkSubmission}
\end{center}

If you are not familiar with Matlab or how Matlab functions work, you can refer to Matlab online documentation for help:
\begin{center}
\url{http://www.mathworks.com/help/matlab/}
\end{center}

In addition, please use built-in Matlab functions rather than external library functions. Without proper reference to external library, auto-grader may fail even if your code runs perfectly on your local machine. Also, please DO NOT include data file in your submission.


\item
\textbf{Collaboration Policy.} 
You are allowed and encouraged to work together. You may discuss the
homework to understand the problem and reach a solution in groups up
to size {\bf two students.} 
You can write {\bf one copy} of solution code per group. However, each group member needs to submit plots separately (can be the same as your collaborator's) in the PDF submitted to canvas.
For the autograder you need to submit {\bf one copy} of your code per group. Be sure to include {\bf your and your collaborator's} pennkey and name in the {\bf group.txt} file. {\bf We will be using automatic checking software to detect blatant copying of other groups' assignments, so,
please, don't do it.}

\end{itemize}

}

\newcommand{\ProgrammingPolicies}[1]{
\paragraph*{Instructions.} 
This is a MATLAB programming assignment. This assignment consists of
multiple parts. For the code part, you should submit your code to autograder via \textbf{turnin}, the instruction of which is listed below; for the pdf, you should submit it through \textbf{Gradescope}. 

We are providing you with codebase / templates / dataset that you will
require for this assignment. Download the file {\tt hw#1\_kit.zip}
from Canvas {\bf before} beginning the assignment.  {\bf Please
  read through the documentation provided in ALL Matlab files before
  starting the assignment.} The instructions for submitting your
homeworks and receiving automatic feedback are online on the wiki:

\begin{center}
\url{http://alliance.seas.upenn.edu/~cis520/wiki/index.php?n=Resources.HomeworkSubmission}
\end{center}

If you are not familiar with Matlab or how Matlab functions work, you can refer to Matlab online documentation for help:
\begin{center}
\url{http://www.mathworks.com/help/matlab/}
\end{center}

In addition, please use built-in Matlab functions rather than external library functions. Without proper reference to external library, auto-grader may fail even if your code runs perfectly on your local machine. Also, please DO NOT include data file in your submission.

\paragraph*{Collaboration.} 
You are allowed and encouraged to work together. You may discuss the
homework to understand the problem and reach a solution in groups up
to size {\bf two students.} Please submit {\bf one copy} of your work. Be sure to include {\bf your and your collaborator's} pennkey and name in the {\bf group.txt} file. {\bf We will be using automatic checking software to detect blatant copying of other groups' assignments, so,
please, don't do it.}
}

%%%% CUSTOM COMMANDS FOR FORMATTING EXAMS/HOMEWORKS
\newcounter{section_points}[section]
\newcounter{header_points}[section]
\newcounter{total_points}

\newcommand{\hpoints}[1]{
  \setcounter{header_points}{#1}
  \textbf{[#1 points]}
}

\newcommand{\points}[1]{
  \addtocounter{section_points}{#1}
  \addtocounter{total_points}{#1}
  \textbf{[#1 
    \ifthenelse{\equal{#1}{1}}
    {point}{points}]}
}

\newcommand{\bpoints}[1]{
  \textbf{[#1 
    \ifthenelse{\equal{#1}{1}}
    {point}{points}]}
}


\newcommand{\point}{\textbf{[1 point]}}

\newboolean{ShowSolutions}
\newcommand{\Mistake}[2]{
  \ifthenelse{\boolean{ShowSolutions}}
  {\paragraph{\bf $\blacksquare$ COMMON MISTAKE #1:} {\sf #2} \bigskip}
  {}
}

\newcommand{\Solution}[2]{
  \ifthenelse{\boolean{ShowSolutions}}
    {
      \paragraph{\bf $\bigstar $ SOLUTION:} { \sf
        #1} \bigskip
    }
    { 
      #2
    } %} \vspace{1.5in}}
}
\newcommand{\out}[1]{}

\newboolean{ShowPointsInfo}

\newcommand{\PointStats}[0]{
  \ifthenelse{\boolean{ShowPointsInfo}}
  {
    \begin{center}

      \begin{tabular}{rl}
        \hline
        Stated Points: & \arabic{header_points} \\
        Section Points: & \arabic{section_points} \\
        Total Points So Far: & \arabic{total_points} \\
        \hline 
        \multicolumn{2}{c}{
          
          \ifthenelse{
            \equal{\value{section_points}}{\value{header_points}}
          }{CORRECT TOTAL}
          {{\bf INCORRECT TOTAL}}
          }
          \\
        \hline
      \end{tabular}
    \end{center}
  }{}
}

\newcounter{blankcount}
\newcommand{\myrepeat}[2] {
\setcounter{blankcount}{1}
\whiledo{\value{blankcount} < #1}{
#2
\addtocounter{blankcount}{1}
}
}

\newcommand{\blank}[1]{\underline{\myrepeat{#1}{\qquad}}}



%================================ PREAMBLE ==================================

%--------- Packages -----------
\usepackage{fullpage}
\usepackage{amssymb}
\usepackage{amsmath}
\usepackage{amsthm}
\usepackage{latexsym}
\usepackage{graphicx}
\usepackage{color}
\usepackage{url}
\usepackage{enumerate}
\usepackage{multirow}

%\usepackage{algorithm,algorithmic}

%----------Spacing-------------
%\setlength{\oddsidemargin}{0.25 in}
%\setlength{\evensidemargin}{-0.25 in}
\setlength{\topmargin}{-0.6 in}
%\setlength{\textwidth}{6.5 in}
%\setlength{\textheight}{9.4 in}
\setlength{\headsep}{0.75 in}
\setlength{\parindent}{0 in}
\setlength{\parskip}{0.1 in}

%----------Header--------------
\newcommand{\assignment}[2]{
   \pagestyle{myheadings}
   \thispagestyle{plain}
   \newpage
   \setcounter{page}{1}
   \noindent
   \begin{center}
   \framebox{
      \vbox{\vspace{2mm}
    \hbox to 6.28in { {\bf CIS 520 Machine Learning} \hfill {\it Due:} #2 }
       \vspace{6mm}
       \hbox to 6.28in { \hfill {\Large Homework #1} \hfill }
       \vspace{6mm}
       \hbox to 6.28in { \hfill }
      \vspace{2mm}}
   }
   \end{center}
   \markboth{CIS 520 Machine Learning, Fall 2017, HW #1}{CIS 520 Machine Learning, Fall 2017, HW #1}
   \vspace*{4mm}
}

%--------Environments----------
\theoremstyle{definition}
\newtheorem{thm}{Theorem}[section]
\newtheorem{lem}[thm]{Lemma}
\newtheorem{prop}[thm]{Proposition}
\newtheorem{cor}[thm]{Corollary}
\newenvironment{pf}{{\noindent\sc Proof. }}{\qed}
\newenvironment{map}{\[\begin{array}{cccc}} {\end{array}\]}

\theoremstyle{definition}
\newtheorem*{defn}{Definition}
\newtheorem{exmp}{Example}
\newtheorem*{prob}{Problem}
\newtheorem*{exer}{Exercise}

\theoremstyle{remark}
\newtheorem*{rem}{Remark}
\newtheorem*{note}{Note}

%---------Definitions----------
\newcommand{\Fig}[1]{Figure~\ref{#1}}
\newcommand{\Sec}[1]{Section~\ref{#1}}
\newcommand{\Tab}[1]{Table~\ref{#1}}
\newcommand{\Tabs}[2]{Tables~\ref{#1}--\ref{#2}}
\newcommand{\Eqn}[1]{Eq.~(\ref{#1})}
\newcommand{\Eqs}[2]{Eqs.~(\ref{#1}-\ref{#2})}
\newcommand{\Lem}[1]{Lemma~\ref{#1}}
\newcommand{\Thm}[1]{Theorem~\ref{#1}}
\newcommand{\Cor}[1]{Corollary~\ref{#1}}
\newcommand{\App}[1]{Appendix~\ref{#1}}
\newcommand{\Def}[1]{Definition~\ref{#1}}
%
\renewcommand{\>}{{\rightarrow}}
\renewcommand{\hat}{\widehat}
\renewcommand{\tilde}{\widetilde}
\newcommand{\half}{\frac{1}{2}}
\newcommand{\R}{{\mathbb R}}
\newcommand{\Z}{{\mathbb Z}}
\newcommand{\N}{{\mathbb N}}
\renewcommand{\P}{{\mathbf P}}
\newcommand{\E}{{\mathbf E}}
\newcommand{\Var}{{\mathbf{Var}}}
\newcommand{\I}{{\mathbf I}}
\newcommand{\1}{{\mathbf 1}}
\newcommand{\0}{{\mathbf 0}}
\renewcommand{\H}{{\mathcal H}}
\newcommand{\F}{{\mathcal F}}
\newcommand{\G}{{\mathcal G}}
\renewcommand{\L}{{\mathcal L}}
\newcommand{\cN}{{\mathcal N}}
\newcommand{\X}{{\mathcal X}}
\newcommand{\Y}{{\mathcal Y}}
\newcommand{\bX}{{\mathbf X}}
\newcommand{\sign}{\textup{\textrm{sign}}}
\newcommand{\er}{\textup{\textrm{er}}}
\newcommand{\abs}{\textup{\textrm{abs}}}
\newcommand{\sq}{\textup{\textrm{sq}}}
\newcommand{\ord}{\textup{\textrm{ord}}}
\newcommand{\zo}{\textup{\textrm{0-1}}}
\newcommand{\hinge}{\textup{\textrm{hinge}}}
\newcommand{\ramp}{\textup{\textrm{ramp}}}
\newcommand{\mar}{\textup{\textrm{margin}}}
\newcommand{\lin}{\textup{\textrm{lin}}}
\newcommand{\poly}{\textup{\textrm{poly}}}
\newcommand{\majority}{\textup{\textrm{majority}}}
\newcommand{\maj}{\textup{\textrm{maj}}}
\newcommand{\co}{\textup{\textrm{co}}}
\newcommand{\agg}{\textup{\textrm{agg}}}
\newcommand{\bad}{\textup{\textrm{bad}}}
\newcommand{\EX}{\textup{\textrm{\textit{EX}}}}
\newcommand{\GD}{\textup{\textrm{{GD}}}}
\newcommand{\EG}{\textup{\textrm{{EG}}}}
%\newcommand{\algorithm}{\textup{\textrm{{algorithm}}}}
\newcommand{\VCdim}{\textup{\textrm{{VCdim}}}}
\newcommand{\VCentropy}{\textup{\textrm{{VC-entropy}}}}
\newcommand{\Pdim}{\textup{\textrm{{Pdim}}}}
\newcommand{\fat}{\textup{\textrm{{fat}}}}
\newcommand{\reject}{\textup{\textrm{{reject}}}}
\newcommand{\rejectsf}{\textup{\textsf{{reject}}}}
\renewcommand{\a}{{\mathbf a}}
\renewcommand{\b}{{\mathbf b}}
\newcommand{\x}{{\mathbf x}}
\newcommand{\y}{{\mathbf y}}
\newcommand{\w}{{\mathbf w}}
\newcommand{\p}{{\mathbf p}}
\newcommand{\q}{{\mathbf q}}
\renewcommand{\r}{{\mathbf r}}
\renewcommand{\u}{{\mathbf u}}
\newcommand{\bloss}{{\boldsymbol \ell}}
\newcommand{\balpha}{{\boldsymbol \alpha}}
\newcommand{\bxi}{{\boldsymbol \xi}}
\newcommand{\bpsi}{{\boldsymbol \psi}}
\newcommand{\btau}{{\boldsymbol \tau}}





